\documentclass[a4paper]{report}
%% Packages %%
\usepackage{hyperref}
\usepackage{color}
\usepackage{graphicx}
\usepackage{titlesec}
\usepackage{listings} 
\usepackage{morefloats}
\usepackage{mathtools}
\usepackage{gensymb}
\usepackage{caption}
\usepackage{subcaption}
\usepackage{url}
\usepackage{adjustbox}

\mathtoolsset{showonlyrefs=true} %% Equations numbers are only shown if they are referred to.

\usepackage[ruled,linesnumbered,resetcount,algochapter]{algorithm2e} %% For producing nice pseudo code algorithms

\usepackage[parfill]{parskip} % line break between graph and no indent
\setlength{\parindent}{1em} % New paragraph indentation

\graphicspath{{../images/}}

%%%%%%%%%%%%%%%%%%%%%%%%%%%%%%%%%%%%%%%%%%%%%%%%%%%%%
%													%
%			User input								%
%			Remeber to fill out all the inputs!		%
%													%
%%%%%%%%%%%%%%%%%%%%%%%%%%%%%%%%%%%%%%%%%%%%%%%%%%%%%
\newcommand{\Title}{Image Segmentation with OpenMPI}
\newcommand{\Author}{Tuan Nguyen Trung}
\newcommand{\Department}{DTU Compute}
\newcommand{\Date}{\today}
\newcommand{\Nameone}{Tuan Nguyen Trung}
\newcommand{\Mailone}{tntr@dtu.dk}


\definecolor{cppcomment}{rgb}{0.133,0.545,0.133}
\definecolor{cppnumbers}{rgb}{0.5,0.5,0.5}

\lstset{
  backgroundcolor=\color{white},   % choose the background color; you must add \usepackage{color} or \usepackage{xcolor}
  basicstyle=\footnotesize,        % the size of the fonts that are used for the code
  breakatwhitespace=false,         % sets if automatic breaks should only happen at whitespace
  breaklines=true,                 % sets automatic line breaking
  captionpos=b,                    % sets the caption-position to bottom
  commentstyle=\color{cppcomment}, % comment style
  escapeinside={*@}{@*},           % if you want to add LaTeX within your code
  extendedchars=true,              % lets you use non-ASCII characters; for 8-bits encodings only, does not work with UTF-8
  frame=single,                    % adds a frame around the code
  keepspaces=true,                 % keeps spaces in text, useful for keeping indentation of code (possibly need columns=flexible)
  keywordstyle=\color{blue},       % keyword style
  numbers=left,                    % where to put the line-numbers; possible values are (none, left, right)
  numbersep=5pt,                   % how far the line-numbers are from the code
  numberstyle=\tiny\color{cppnumbers},
  rulecolor=\color{black},         % if not set, the frame-color may be changed on line-breaks within not-black text (e.g. comments (green here))
  showspaces=false,                % show spaces everywhere adding particular underscores; it overrides 'showstringspaces'
  showstringspaces=false,          % underline spaces within strings only
  showtabs=false,                  % show tabs within strings adding particular underscores
  stepnumber=1,                    % the step between two line-numbers. If it's 1, each line will be numbered
  stringstyle=\color{blue},        % string literal style
  tabsize=2                      % sets default tabsize to 2 spaces
}
\hypersetup{
    bookmarks=true,         % show bookmarks bar?
    unicode=false,          % non-Latin characters in Acrobat’s bookmarks
    pdfborder={0 0 0 [3 3]},% removes border around links.
    pdffitwindow=false,     % window fit to page when opened
    pdfstartview={FitB},    % fits the width of the page to the window
    pdftitle={\Title},       % title
    pdfauthor={\Author},     % author
    pdfnewwindow=true,      % links in new window
    colorlinks=false,       % false: boxed links; true: colored links
    linkcolor=red,          % color of internal links
    citecolor=green,        % color of links to bibliography
    filecolor=magenta,      % color of file links
    urlcolor=cyan           % color of external links
}

\newcommand{\HRule}{\rule{\linewidth}{0.5mm}} % new command
\titleformat{\chapter}[hang]{\normalfont\huge\bfseries}{\thechapter}{1em}{}

\begin{document}
\begin{titlepage}
 	\begin{center}
		% Upper part of the page
		%\includegraphics[scale=1]{dtulogo}\\[2.5cm]
		%\vspace*{\fill}
		\noindent \textsc{\Large \Department }\\[0.2cm]
 		% Title
		\HRule \\[0.4cm]
		{ \huge \bfseries \Title}\\[0.4cm]
 		\HRule \\[1 cm]
 		% Author and supervisor
		\begin{minipage}{0.6\textwidth}
			\large
		%	\emph{Author:}	\\
		%	\newline\newline 
			\Nameone, \Mailone
			\newline \newline \newline

			 
		%	\rule{140pt}{0.5pt}\\
		%	\newline
		\end{minipage}		
		\vfill
 	\end{center}
	\begin{center}
		{\large \Date}
	\end{center}
\end{titlepage}

\newpage
\input{summary.tex}
\newpage
\tableofcontents
\newpage
\listoffigures
\newpage
%%%%%%%%%%%%%%%%%%%% Start text inputs


%\begin{abstract}
%This paper studies parallel algorithm for active contour model proposed by Chan and Vese \cite{Chan2001}. The purpose is to segment regions of image by minimizing Mumford-Shah model. The paper studies serial algorithm, written by Getreuer \cite{Getreuer2012}, and improve this version using Open MPI.
%\end{abstract}


\section{Active contour without edge discretization}

Mumford-Shah model:

\begin{equation}
E = \mu \text{ length}(C) 
+ \lambda\int_{\Omega}(f(x) - u(x))^2 dx
+ \int_{\Omega \\ C}|\nabla u(x)|^2 dx
\end{equation}
where $\Omega$ denotes the image region. $C$ denotes the contour of the region, which is a parametric curve. $f(x)$ is the intensity of image pixels. $u(x)$ is the piecewise function that approximate the image. $\mu$ and $\lambda$ is constant.

$u(x)$ is represented using level set function: $\Phi$ on domain of the image. Level set function is discretized to same size of the image. Final Chan-Vese model for level set function of minimal Mumford-Shah

\begin{equation}
    \left\{ 
        \begin{aligned}
        & \frac{\partial \phi}{\partial t} = 
                \delta_{\epsilon} \left[
                \mu \text{ div} \left( \frac{\nabla \phi}{|\nabla \phi|} \right) - 
                \nu - 
                \lambda_1 (f-c_1)^2 +
                \lambda_2 (f-c_2)^2
                \right] \quad \text{ in } \Omega\\
        & \frac{\delta_{\epsilon}(\phi)}{|\nabla \phi|} 
            \frac{\partial \phi}{\partial \overrightarrow{n}}
            = 0
            \qquad \text{ on } \partial \Omega
        \end{aligned}
    \right.
\end{equation}

\begin{equation}
    \begin{aligned}
        &	c_1 \text{ and } c_2 \text{ is the mean intensity of the region inside and outside} \\
        & 	\overrightarrow{n} 
            \text{ is outward normal of the image boundary} \\
        & 	\phi 
            \text{ is the level set function} \\
        & 	\delta_{\epsilon}(\phi) = \frac{\epsilon}{\pi (\epsilon^2 + \phi^2)} 
            \text{ is Dirac function. Derivative of heaviside funstion} H_{\epsilon}(t) \\
        & \nu,\ \eta,\ \lambda_1 \text{ and } \lambda_2 \text{ are constant}
    \end{aligned}
\end{equation}

For implementation, level set function is discretized to a matrix same size of the image. The final function to compute $\phi$

\begin{equation}
    \begin{aligned}
            \frac{\partial \phi_{i, j}}{\partial t} = 
            \delta_{\epsilon}(\phi_{i,j}) \left[ \right.
                    & \left( A_{i,j}(\phi_{i+1, j} - \phi_{i,j})
                            - A_{i-1,j}(\phi_{i, j} - \phi_{i,j-1}) \right) \\
                    & + \left( B_{i, j}(\phi_{i, j+1} - \phi_{i, j}) - 
                             B_{i, j-1}(\phi_{i, j} - \phi_{i, j-1}) \right) \\
                    & - \nu - 
                        \lambda_1(f_{i,j} - c_1)^2 +
                        \lambda_2(f_{i, j} - c_2)^2 
                        \left. \right]
    \end{aligned}
    \label{eq:phi}
\end{equation}
where $A$ and $B$ are matrix same size with $\phi$.

\begin{equation}
    A_{i, j} = \frac{\mu}
                {\sqrt{\eta^2 
                       + (\nabla_x^+ \phi_{i, j} )^2
                       + (\nabla_y^0 \phi_{i, j} )^2 }
                }
    \label{eq:a}
\end{equation}

\begin{equation}
    B_{i, j} = \frac{\nu}
                    {\sqrt{\eta^2 
                            + (\nabla_x^0 \phi_{i, j} )^2
                            + (\nabla_y^+ \phi_{i, j} )^2 } 
                    }
    \label{eq:b}
\end{equation}



$\Delta \phi(i, j)$ denotes the difference of the level set function. There are forward, backward and central difference

\begin{equation}
    \begin{aligned}
        \nabla_x^- \phi(i, j) &= \phi_{i, j} - \phi_{i-1, j} \\
        \nabla_x^+ \phi(i, j) &= \phi_{i+1, j} - \phi_{i, j} \\
        \nabla_x^0 \phi(i, j) &= (\phi_{i+1, j} - \phi_{i-1, j} ) / 2
    \end{aligned}
\end{equation}

\section{Serial algorithm}
Sample code is written by Getreuer \cite{Getreuer2012}. The algorithm is described in algorithm \ref{alg:chan-vese}.The sample results is shown in fig. \ref{fig:seg} and fig. \ref*{fig:prog}

\begin{algorithm}[hb]
    \DontPrintSemicolon
    \KwData{Image $I$}
    Initialize $\phi$ \;
    \For{ n = 1, .. }{
        Compute $c_1$ and $c_2$ \;
        \Begin(Compute $\phi^{n+1}$){
            Compute $A$ and $B$ \eqref{eq:a}, \eqref{eq:b}\;
            Compute $\frac{\partial \phi}{\partial t}$ \eqref{eq:phi}\;
            Compute $\phi^{n+1} = \phi^n + \frac{\partial \phi}{\partial t} \Delta t$
        }
        
        \textbf{if} $|\phi^{n+1} - \phi^n| < thres$ \textbf{then break} 
    }

    \caption{Numerical implementation}
    \label{alg:chan-vese}
\end{algorithm}



\begin{figure}[ht]
    \centering
    \begin{tabular}{cc}
        \includegraphics[scale=0.2]{serial_1/fox} & \includegraphics[scale=0.2]{serial_1/final} \\
        \small a) Original image & \small b) Segmented image
    \end{tabular}
    \caption{Image segmentation}
    \label{fig:seg}
\end{figure}

\begin{figure}[!htb]
    \centering
    \newcommand{\imgFoxProg}[1]{\includegraphics[trim=3cm 1cm 3cm 1cm, clip, scale=0.12]{serial_1/#1}}
    \imgFoxProg{1} \imgFoxProg{2} \imgFoxProg{3} \imgFoxProg{4} 
    \caption{Progress of segmentation}
    \label{fig:prog}
\end{figure}

\section{Parallel algorithm}

Dependency in algorithm \ref{alg:chan-vese}:
\begin{itemize}
    \item Computation of average intensity $c_1$ and $c_2$ are collective sum
    \item Computation of $\phi(i, j)$ depends on neighbour pixels
    \item Computation of error $|\phi^{n+1} - \phi^n|$ is collective sum
\end{itemize}

\noindent The algorithm for parallel MPI is shown in algorithm \ref{alg:chan-vese-mpi}. We need to discretize the image and the domain of level set funstion 

\begin{algorithm}[hb]
    \DontPrintSemicolon
    \KwData{Partial image $f_i$}
    Initialize partial level set function $\phi_i$ \;
    \For{ n = 1, .. }{
        Compute $c_1$ and $c_2$ from $f_i$ locally\;
        MPI: Collective sum of $c_1$ and $c_2$ \;\;
        
        \Begin(Compute $\phi_i^{n+1}$ locally){
            Compute $A_i$ and $B_i$ from from $f_i$ and $\phi_i^n$ \eqref{eq:a} \eqref{eq:b}\;
            Compute $\frac{\partial \phi_i}{\partial t}$ from  \eqref{eq:phi}\;
            Compute $\phi_i^{n+1} = \phi_i^n + \frac{\partial \phi_i}{\partial t} \Delta t$
        }\;
        
        Compute error $|\phi_i^{n+1} - \phi_i^n|$ locally \;
        MPI: Collective total error $e$ \;
        \textbf{if} $e < thres$ \textbf{then stop}  \; \;
        
        MPI: Exchange boundaries \;
        
    }
    
    \caption{OpenMPI algorithm}
    \label{alg:chan-vese-mpi}
\end{algorithm}


\section{Algorithm}
Dependency in algorithm \ref{alg:chan-vese}:
\begin{itemize}
    \item Computation of average intensity $c_1$ and $c_2$ are collective sum
    \item Computation of $\phi(i, j)$ depends on neighbour pixels
    \item Computation of error $|\phi^{n+1} - \phi^n|$ is collective sum
\end{itemize}

\noindent The algorithm for parallel MPI is shown in algorithm \ref{alg:chan-vese-mpi}. We need to discretize the image and the domain of level set funstion 

\begin{algorithm}[!htb]
    \DontPrintSemicolon
    \KwData{Partial image $f_i$}
    Load partial image in each process
    Initialize partial level set function $\phi_i$ \;
    \For{ n = 1, .. }{
        Compute $c_1$ and $c_2$ from $f_i$ locally\;
        MPI: Collective sum of $c_1$ and $c_2$ \;\;
        
        \Begin(Compute $\phi_i^{n+1}$ locally){
            Compute $A_i$ and $B_i$ from from $f_i$ and $\phi_i^n$ \eqref{eq:a} \eqref{eq:b}\;
            Compute $\frac{\partial \phi_i}{\partial t}$ from  \eqref{eq:phi}\;
            Compute $\phi_i^{n+1} = \phi_i^n + \frac{\partial \phi_i}{\partial t} \Delta t$
        }\;
        
        Compute error $|\phi_i^{n+1} - \phi_i^n|$ locally \;
        MPI: Collective total error $e$ \;
        \textbf{if} $e < thres$ \textbf{then stop}  \; \;
        
        MPI: Exchange boundaries \;
        
    }
    
    \caption{OpenMPI algorithm}
    \label{alg:chan-vese-mpi}
\end{algorithm}

\section{Implementation}
\textbf{Data discretization:} The image domain is discretized to 2-dimensional grid. Each grid element is a square area size $\text{block\_size} \times \text{block\_size}$, and it corresponds with a MPI process. Because we need the neighbour data, the buffer data in each block is 2 pixel larger than the block size.

\begin{figure}[!htb]
    \centering
    \includegraphics[trim = 20cm 20cm 20cm 15cm, clip, scale = 0.04]{block_demo}
    \caption{Discretization of image domain}
    \label{fig:dis}
\end{figure}

The blocks are stored by row order.

\noindent\textbf{Init level set function:} Level set is a two-dimensional function $\phi = \phi(x,y)$. We discretize this function to same size with the image $\phi = \phi_{i,j}$. We use the same discretization for the image and the level set. A simple initialization of $\phi$ is
\begin{equation}
    \phi(i,j) = \sin\frac{i\pi}{5} * \sin\frac{j\pi}{5}
\end{equation}

The initialized level set function look like:
\begin{figure}[!htb]
    \centering
    \includegraphics[scale=0.07]{phi_init}
    \caption{Initialization of level set function (Zoom is $50 \times 30$ pixels)}
\end{figure}

\noindent\textbf{Intensity average:} To compute the average intensity of inside region and out side region, we compute total intensity in reach block and use MPI\_reduce to sum the total intensity. Sample code is in appendix \ref{apdx:regionAverage}.

\noindent\textbf{Update level set function:} As explain in chapter \ref{chap:chan}.

\noindent\textbf{Exchange boundary:} Boundary of level set function need to be update after each iteration. We use MPI\_Send and MPI\_Recv to transfer the data. Code is in appendix 

\noindent\textbf{Gather level set function:} To obtain full level set, we have to collect parts from all processes. Code in appendix \ref{apdx:gatherphi}

\section{Results}

\newcommand{\figPhi}[1]{\includegraphics[scale = 0.1]{phi_1/#1}}
\begin{figure}[!htb]
    \centering
    \begin{tabular}{cc}
        \figPhi{1} & \figPhi{10} \\
        iter 1 & iter 10 \\
        \figPhi{15} & \figPhi{20} \\
        iter 15 & iter 20 \\
        \figPhi{40} & \figPhi{100} \\
        iter 40 & iter 100
    \end{tabular}
     \caption{Segmented image progress}
\end{figure}
\input{tunning.tex}
\input{conclusion.tex}


%%%%%%%%%%%%%%%%%%% Citation
\bibliographystyle{plain}
\bibliography{citation}

%%%%%%%%%%%%%%%%%%% Start appendix
\newpage
\appendix
\chapter{Code}

Some defined types:

\newcommand{\ck}[1]{\textcolor{blue}{#1}}
\begin{itemize}
    \item \ck{num}: \ck{float} or \ck{double} (Depend of precision).
\end{itemize}

\section{Region average}
\label{apdx:regionAverage}
\begin{lstlisting}[caption=Region average]
void region_average(num *c1, num *c2){
    num sum1 = 0.0, sum2 = 0.0;
    long count1 = 0, count2 = 0;
    
    for (int x = 0; x < g.active_size_x; x++) {
        for (int y = 0; y < g.active_size_y; y++) {
            if (get_phi_data(x, y) >= 0) {
                count1++;
                sum1 += get_sub_image_data(x, y);
            }else{
                count2++;
                sum2 += get_sub_image_data(x, y);
            }
        }
    }
    
    num total_sum1 = 0.0, total_sum2 = 0.0;
    long total_count1 = 0, total_count2 = 0;
    
    MPI_Allreduce(&sum1, &total_sum1, 1, 
                    MPI_NUM, MPI_SUM, MPI_COMM_WORLD);
    MPI_Allreduce(&sum2, &total_sum2, 1, 
                    MPI_NUM, MPI_SUM, MPI_COMM_WORLD);
    MPI_Allreduce(&count1, &total_count1, 1, 
                    MPI_LONG, MPI_SUM, MPI_COMM_WORLD);
    MPI_Allreduce(&count2, &total_count2, 1, 
                    MPI_LONG, MPI_SUM, MPI_COMM_WORLD);
    
    *c1 = total_sum1 / (num)total_count1;
    *c2 = total_sum2 / (num)total_count2;
}
\end{lstlisting}

\section{Gathering level set function}
\label{apdx:gatherphi}
\begin{lstlisting}[caption=Gather phi]
void gather_phi(){
    num * main_domain = malloc(...);
    for (int i = 0; i < active_size_y; i++) {
        memcpy(main_domain + i*block_size <- 
                -- phi  + local_array_idx(0, i));
    }
    
    if (rank != main_proc) { // Send
        MPI_Send(main_domain,
                block_size * block_size,
                MPI_NUM,
                main_proc,
                GATHER_PHI,
                MPI_COMM_WORLD);
    } /* if (rank != main_proc) */
    
    if (rank == main_proc) { // receive
        num * phi_total = malloc(...);
        for (int blx = 0; blx < bl_dim_x; blx ++) {
            for (int bly = 0; bly < bl_dim_y; bly++) {
                int idx = bly*bl_dim_x + blx;
                num * receive_domain = malloc(size);
                
                if (idx == main_proc) {
                    memcpy(receive_domain <- main_domain);
                }
                else{
                    MPI_Status stat;
                    MPI_Recv(receive_domain,
                    block_size * block_size,
                    MPI_NUM,
                    idx,
                    GATHER_PHI,
                    MPI_COMM_WORLD,
                    &stat);
                }
                
                // length_y is valid height of the block
                for (int y = 0; y < length_y; y++) {
                    int gx = blx*block_size;
                    int gy = bly*block_size + y;
                    memcpy(phi_total + gy*image_width+gx,
                    receive_domain + y*block_size,
                    length_x*sizeof(num));
                }
                
                free(receive_domain);
            }
        }
    } /* if (rank == main_proc) */
    
    free(main_domain);
}
\end{lstlisting}

\section{Transfer boundary}
\label{apdx:tranbound}

\begin{lstlisting}[caption= Transfer boundary]
void exchange_boundary(){
    // exchange left boundary with bl_idx_x - 1
    if (bl_idx_x > 0) {
        num * left = malloc(block_size*sizeof(num));
        for (int j = 0; j < block_size; j++) {
            left[j] = get_phi_data(-1, j);
        }

        MPI_Send(left,
                block_size,
                MPI_NUM,
                block_idx(bl_idx_x-1, bl_idx_y),
                EXCHANGE_BOUND,
                MPI_COMM_WORLD);
        
        // Receive
        MPI_Status stat;
        num * right_recv = malloc(block_size*sizeof(num));
        MPI_Recv(right_recv, 
                 block_size, 
                 MPI_NUM, 
                 block_idx(bl_idx_x-1, bl_idx_y),
                 EXCHANGE_BOUND, 
                 MPI_COMM_WORLD, 
                 &stat);
        for (int j = 0; j < block_size; j++) {
            set_phi_data(-1, j, right_recv[j]);
        }
        
        free(left);
        free(right_recv);
    }
   
    /* Do the same for bottom, right and top of the block */
}
\end{lstlisting}

%%%%%%%%%%%%%%%%%%% End
\end{document}