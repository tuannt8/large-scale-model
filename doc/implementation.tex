
\section{Algorithm}
Dependency in algorithm \ref{alg:chan-vese}:
\begin{itemize}
    \item Computation of average intensity $c_1$ and $c_2$ are collective sum
    \item Computation of $\phi(i, j)$ depends on neighbour pixels
    \item Computation of error $|\phi^{n+1} - \phi^n|$ is collective sum
\end{itemize}

\noindent The algorithm for parallel MPI is shown in algorithm \ref{alg:chan-vese-mpi}. We need to discretize the image and the domain of level set funstion 

\begin{algorithm}[!htb]
    \DontPrintSemicolon
    \KwData{Partial image $f_i$}
    Load partial image in each process
    Initialize partial level set function $\phi_i$ \;
    \For{ n = 1, .. }{
        Compute $c_1$ and $c_2$ from $f_i$ locally\;
        MPI: Collective sum of $c_1$ and $c_2$ \;\;
        
        \Begin(Compute $\phi_i^{n+1}$ locally){
            Compute $A_i$ and $B_i$ from from $f_i$ and $\phi_i^n$ \eqref{eq:a} \eqref{eq:b}\;
            Compute $\frac{\partial \phi_i}{\partial t}$ from  \eqref{eq:phi}\;
            Compute $\phi_i^{n+1} = \phi_i^n + \frac{\partial \phi_i}{\partial t} \Delta t$
        }\;
        
        Compute error $|\phi_i^{n+1} - \phi_i^n|$ locally \;
        MPI: Collective total error $e$ \;
        \textbf{if} $e < thres$ \textbf{then stop}  \; \;
        
        MPI: Exchange boundaries \;
        
    }
    
    \caption{OpenMPI algorithm}
    \label{alg:chan-vese-mpi}
\end{algorithm}

\section{Implementation}
\textbf{Data discretization:} The image domain is discretized to 2-dimensional grid. Each grid element is a square area size $\text{block\_size} \times \text{block\_size}$, and it corresponds with a MPI process. Because we need the neighbour data, the buffer data in each block is 2 pixel larger than the block size.

\begin{figure}[!htb]
    \centering
    \includegraphics[trim = 20cm 20cm 20cm 15cm, clip, scale = 0.04]{block_demo}
    \caption{Discretization of image domain}
    \label{fig:dis}
\end{figure}

The blocks are stored by row order.

\noindent\textbf{Init level set function:} Level set is a two-dimensional function $\phi = \phi(x,y)$. We discretize this function to same size with the image $\phi = \phi_{i,j}$. We use the same discretization for the image and the level set. A simple initialization of $\phi$ is
\begin{equation}
    \phi(i,j) = \sin\frac{i\pi}{5} * \sin\frac{j\pi}{5}
\end{equation}

The initialized level set function look like:
\begin{figure}[!htb]
    \centering
    \includegraphics[scale=0.07]{phi_init}
    \caption{Initialization of level set function (Zoom is $50 \times 30$ pixels)}
\end{figure}

\noindent\textbf{Intensity average:} To compute the average intensity of inside region and out side region, we compute total intensity in reach block and use MPI\_reduce to sum the total intensity. Sample code is in appendix \ref{apdx:regionAverage}.

\noindent\textbf{Update level set function:} As explain in chapter \ref{chap:chan}.

\noindent\textbf{Exchange boundary:} Boundary of level set function need to be update after each iteration. We use MPI\_Send and MPI\_Recv to transfer the data. Code is in appendix 

\noindent\textbf{Gather level set function:} To obtain full level set, we have to collect parts from all processes. Code in appendix \ref{apdx:gatherphi}

\section{Results}

\newcommand{\figPhi}[1]{\includegraphics[scale = 0.1]{phi_1/#1}}
\begin{figure}[!htb]
    \centering
    \begin{tabular}{cc}
        \figPhi{1} & \figPhi{10} \\
        iter 1 & iter 10 \\
        \figPhi{15} & \figPhi{20} \\
        iter 15 & iter 20 \\
        \figPhi{40} & \figPhi{100} \\
        iter 40 & iter 100
    \end{tabular}
     \caption{Segmented image progress}
\end{figure}